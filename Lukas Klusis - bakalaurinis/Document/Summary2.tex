\documentclass[]{article}
\usepackage[a4paper, total={6in, 8in}]{geometry}
%opening
\title{Summary of Bachelor's thesis}
\author{Lukas Klusis}
\date{}

\begin{document}
	
	\sloppy
	
	\pagenumbering{Roman}
	\thispagestyle{empty}
	\begin{center}
		
		
		\LARGE{Summary of Bachelor's thesis\\}
		\vspace{0.2cm}
		\LARGE{ Vision-Based Localization for Unmanned Aerial Vehicles}\\
		\vspace{0.5cm}		
		\textbf{\large Lukas Klusis} \\
		\vspace{1cm}	
		
		\large{VILNIUS UNIVERSITY\\} 
		\vspace{0.2cm}
		\large{FACULTY OF MATHEMATICS AND INFORMATICS\\}
		\vspace{2cm}				
	\end{center}
	
	\setlength{\parindent}{4em}
	\setlength{\parskip}{1em}
	\paragraph{}
	\large
	Many navigation systems are strongly dependent on the Global Positioning System (GPS), however as using any other external sensor, reliance on GPS do not give full autonomy and reliability. As an example GPS signal can be spoofed or jammed, this significantly reduces possibilities of Unmanned aerial vehicles (UAV) which tasks often are held in urban areas or conflict zones. In this work it is analysed possible alternatives to the GPS system which would use video camera as a primary sensor and proposes how to implement vision-based navigation system. In this proposal video footage from the camera is registered to the geographical map in order to assess UAV coordinates in WGS-84 coordinate system. In addition these coordinates are combined with data from inertial navigation system. These decisions implements a fully-fledged navigation system. Experimental studies were conducted where simulations were played using data from real flights. Study showed that the system is accurate enough to fulfill practical needs of UAV navigation and the error does not accumulate over time, so the system can change the GPS to ensure complete autonomy of the aircraft.
	
	\begin{center}
		\vspace{4cm}	
		Vilnius \ \ 2015
	\end{center}	
	
	\clearpage
	
\end{document}
